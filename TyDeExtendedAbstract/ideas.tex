\section{Main Ideas}

Here we define the syntax and type-directed judgmental equalities, and also demonstrate the close connection between the type-directed equality rules and the bidirectional typing rules.
We first present both the syntax and type-directed equalities for a small dependent type theory in which the two equalities differ only in presentation.
Then, we present an extension to the theory with two new types.
We add the unit type to show that the type-directed equality supports more rules than the syntax-directed equality.
And we add dependent pairs to further demonstrate the connection between the equality and typing rules.
% Then, to show that the type-directed equality supports more rules than the syntax-directed equality, we extend the theory with the unit type.

Our purpose here is to give the reader a rigorous understanding of what these two equalities are and how they are implemented in smalltt.
Therefore, we present the theory in a form very close to that of both smalltt implementations, but we don't prove any theorems about this theory.
That being said, the theory we present is very similar to that proposed in \citet{altenkirch2010}.

\subsection{Syntax and Semantics}

\begin{figure}[!htb]
  \begin{displaymath}
    \begin{array}{lrcll}
      \text{Expressions} & e & \bnfdef & x & \bnfcomment{variables} \\
      & & \bnfalt & \lamE{x}{e} & \bnfcomment{function literal} \\
      & & \bnfalt & \appE{e_f}{e_a} & \bnfcomment{function application} \\
      & & \bnfalt & \piE{x}{e_d}{e_c} & \bnfcomment{dependent function type} \\
      & & \bnfalt & \univE & \bnfcomment{type universe} \\
      & & \bnfalt & \annE{e}{e_t} & \bnfcomment{type annotation} \\
      \\
      \text{Normal Forms} & n & \bnfdef & \lamE{x}{e} & \\
      & & \bnfalt & \piE{x}{n_d}{e_c} & \\
      & & \bnfalt & \univE & \\
      & & \bnfalt & \nu & \\
      \\
      \text{Neutral Terms} & \nu & \bnfdef & x & \\
      & & \bnfalt & \appE{\nu_f}{n_a} & \\
      \\
      \text{Type Contexts} & \Gamma & \bnfdef & \cdot & \\
      & & \bnfalt & \Gamma, x : n & \\
    \end{array}
  \end{displaymath}
  \caption{Syntax}
  \label{fig:base-syntax}
\end{figure}

Before defining the two kinds of judgmental equality, we must first define the theory whose terms we want to judge equal.
The syntax is presented in \autoref{fig:base-syntax}.
It consists of expressions $e$ which are either variables $x$, function introduction or elimination forms, dependent function types, the type universe, or a type annotation.
We also define a subset of expressions called normal forms.
These are expressions which only contain redexes that are inside a binding form.
We will use these to represent the types during type checking since we don't need to reduce them to check if they are type constructors.
For example, if we used expressions to represent types, we could have $\appE{(\lamE{x}{\piE{y}{\unittE}{\unittE}})}{\unitE}$ which we would need to reduce to $\piE{y}{\unittE}{\unittE}$ to know that it is a type constructor.

\begin{figure}[!htb]
  \fbox{$\steps{e}{n}$}
  \begin{mathpar}
    \inferrule*[left=Var$\Downarrow$]
    {
    }
    { \steps{x}{x}
    }

    \inferrule*[left=$\Pi$-I$\Downarrow$]
    {
    }
    { \steps{\lamE{x}{e}}{\lamE{x}{e}}
    }

    \inferrule*[left=$\beta\Downarrow$]
    { \steps{e_f}{\lamE{x}{e_b}} \\
      \steps{\subst{e_b}{x}{e_a}}{n_b}
    }
    { \steps{\appE{e_f}{e_a}}{n_b}
    }

    \inferrule*[left=$\Pi$-E$\Downarrow$]
    { \steps{e_f}{\nu_f} \\
      \steps{e_a}{n_a}
    }
    { \steps{\appE{e_f}{e_a}}{\appE{\nu_f}{n_a}}
    }

    \inferrule*[left=$\Pi$-T$\Downarrow$]
    { \steps{e_d}{n_d}
    }
    { \steps{\piE{x}{n_d}{e_c}}{\piE{x}{n_d}{e_c}}
    }

    \inferrule*[left=$\univE$-T$\Downarrow$]
    {
    }
    { \steps{\univE}{\univE}
    }

    \inferrule*[left=Ann$\Downarrow$]
    { \steps{e}{n}
    }
    { \steps{\annE{e}{e_t}}{n}
    }
  \end{mathpar}
  \caption{Big Step Semantics}
  \label{fig:base-big-step}
\end{figure}

In \autoref{fig:base-big-step} we define the operational semantics for the theory.
In addition to being a semantics for the theory, we can view these rules as a method for turning an expression into its corresponding normal form.

\begin{figure}[!htb]
  \fbox{$\checkJ{\Gamma}{e}{n_t}$} \\
  \fbox{$\synthJ{\Gamma}{e}{n_t}$}
  \begin{mathpar}
    \inferrule*[left=Var]
    { (x : n) \in \Gamma
    }
    { \synthJ{\Gamma}{x}{n}
    }

    \inferrule*[left=Conv]
    { \synthJ{\Gamma}{e}{n_t} \\
      \tyEqJ{\Gamma}{n_t}{n_s}{\univE}
    }
    { \checkJ{\Gamma}{e}{n_s}
    }

    \inferrule*[left=Ann]
    { \checkJ{\Gamma}{e_t}{\univE} \\
      \steps{e_t}{n_t} \\
      \checkJ{\Gamma}{e}{n_t}
    }
    { \synthJ{\Gamma}{\annE{e}{e_t}}{e_t}
    }

    \inferrule*[left=$\Pi$-I]
    { \freshJ{\Gamma}{z} \\
      \steps{\subst{e_c}{y}{z}}{n_c} \\
      \checkJ{\Gamma, z : n_d}{\subst{e}{x}{z}}{n_c}
    }
    { \checkJ{\Gamma}{\lamE{x}{e}}{\piE{y}{n_d}{e_c}}
    }

    \inferrule*[left=$\Pi$-E]
    { \synthJ{\Gamma}{e_f}{\piE{x}{n_d}{e_c}} \\
      \checkJ{\Gamma}{e_a}{n_d} \\
      \steps{\subst{e_c}{x}{e_a}}{n_c}
    }
    { \synthJ{\Gamma}{\appE{e_f}{e_a}}{n_c}
    }

    \inferrule*[left=$\Pi$-T]
    { \checkJ{\Gamma}{e_d}{\univE} \\
      \checkJ{\Gamma, x : e_d}{e_c}{\univE}
    }
    { \checkJ{\Gamma}{\piE{x}{e_d}{e_c}}{\univE}
    }

    \inferrule*[left=$\univE$-in-$\univE$]
    {
    }
    { \checkJ{\Gamma}{\univE}{\univE}
    }
  \end{mathpar}
  \caption{Typing Rules}
  \label{fig:base-typing-rules}
\end{figure}

Finally, we present the typing rules for the theory in a bidirectional style \citep{Dunfield2021}.
The check judgment $\checkJ{\Gamma}{e}{e_t}$ means that the term $e$ checks against the type $n_t$ under the context $\Gamma$.
It is formulated such that it can be decided when given concrete terms for $\Gamma$, $e$, and $n_t$.
The synth judgment $\synthJ{\Gamma}{e}{n_t}$ means that the type $n_t$ can be inferred for the term $e$ under the context $\Gamma$.
It is formulated such that given the concrete terms for $\Gamma$ and $e$, it can be decided if there exists an $n_t$ such that the judgment holds.

Because of the $\univE$-in-$\univE$ rule our theory is inconsistent, but we still add it for two reasons.
First of all, the problem of adding a stratified hierarchy of universes to make the theory consistent has been studied extensively elsewhere and is orthogonal to our concerns.
Second of all, smalltt uses this rule and so adding it ensures our presentation of the theory remains as close to it as possible.

The $\Pi$-I rule uses the $\freshJ{\Gamma}{z}$ judgment to specify that the variable $z$ must not appear in the context $\Gamma$.
It's conventional to assert that variables are fresh in a context, and so we do not define this judgment here.

The Conv rule uses the typed equality judgment $\tyEqJ{\Gamma}{n_a}{n_b}{n_t}$ which we deliberately haven't defined yet.
In the next two subsections we will present two different versions of this judgment: the syntax-directed version which can be decided by only considering $\Gamma$, $n_a$, and $n_b$, and the type-directed version which also needs to consider $n_t$.

\subsection{Syntax Directed Equality}

\begin{figure}[!htb]
  \begin{displaymath}
    \begin{array}{lrcll}
      \text{Scope} & \Theta & \bnfdef & \cdot & \\
      & & \bnfalt & \Theta, x & \\
    \end{array}
  \end{displaymath}
  \fbox{$\tyEqJ{\Gamma}{n_a}{n_b}{n_t}$}
  \fbox{$\stxEqJ{\Theta}{n_a}{n_b}$}
  $\text{strip-types} : \Gamma \to \Theta$ \\
  $\text{strip-types}(\cdot) = \cdot$ \\
  $\text{strip-types}(\Gamma, x : n_t) = \text{strip-types}(\Gamma), x$ \\
  \begin{mathpar}
    \inferrule*[left=Syn]
    { \stxEqJ{\text{strip-types}(\Gamma)}{n_a}{n_b}
    }
    { \tyEqJ{\Gamma}{n_a}{n_b}{n_t}
    }

    \label{rule:stx-eta-l}
    \inferrule*[left=$\eta$-L]
    { \freshJ{\Theta}{y} \\
      \steps{\appE{(\lamE{x}{e_b})}{y}}{n_b} \\
      \steps{\appE{n}{y}}{n_a} \\
      \stxEqJ{\Theta, y}{n_b}{n_a}
    }
    { \stxEqJ{\Theta}{\lamE{x}{e_b}}{n}
    }

    \label{rule:stx-eta-r}
    \inferrule*[left=$\eta$-R]
    { \freshJ{\Theta}{y} \\
      \steps{\appE{(\lamE{x}{e_b})}{y}}{n_b} \\
      \steps{\appE{n}{y}}{n_a} \\
      \stxEqJ{\Theta, y}{n_b}{n_a}
    }
    { \stxEqJ{\Theta}{n}{\lamE{x}{e_b}}
    }

    \label{rule:stx-pi-t}
    \inferrule*[left=$\Pi{=}$]
    { \stxEqJ{\Theta}{n_d}{n_d'} \\
      \freshJ{\Theta}{y} \\
      \steps{\subst{e_c}{x}{y}}{n_c} \\
      \steps{\subst{e_c'}{x'}{y}}{n_c'} \\
      \stxEqJ{\Theta, y}{n_c}{n_c'}
    }
    { \stxEqJ{\Theta}{\piE{x}{n_d}{e_c}}{\piE{x'}{n_d'}{e_c'}}
    }

    \inferrule*[left=$\univE{=}$]
    {
    }
    { \stxEqJ{\Theta}{\univE}{\univE}
    }

    \inferrule*[left=Var${=}$]
    {
    }
    { \stxEqJ{\Theta}{x}{x}
    }

    \inferrule*[left=App${=}$]
    { \stxEqJ{\Theta}{\nu_f}{\nu_f'} \\
      \stxEqJ{\Theta}{n_a}{n_a'}
    }
    { \stxEqJ{\Theta}{\appE{\nu_f}{n_a}}{\appE{\nu_f'}{n_a'}}
    }
  \end{mathpar}
  \caption{Syntax Directed Equality}
  \label{fig:base-syntax-directed-equality}
\end{figure}

In \autoref{fig:base-syntax-directed-equality} we define the typed equality judgment by appealing to the syntax-directed equality judgment $\stxEqJ{\Theta}{n_a}{n_b}$.
The $\pi$=, $\eta$-l, and $\eta$-r rules require adding fresh variables to the context.
However, we don't know what type these new variables should be given in the $\eta$ cases.
Therefore we use $\Theta$ in the syntax-directed equality judgment instead of $\Gamma$, and use strip-types to turn a type context into a scope.

Since we are already given terms in normal form, we don't need to consider $\beta$ equalities until we get to a binding form.
In these cases, specifically the $\pi$=, $\eta$-l, and $\eta$-r rules, we use the big step semantics from \autoref{fig:base-big-step} to reduce all expressions to their normal form before continuing.

In the $\eta$-l and $\eta$-r rules we use a trick to decide the $\eta$ rule for dependent function from only the syntax.
In a normal form, there are only two cases which could be given a function type.
Either the normal form is a function literal, or it is a neutral term.
If both sides of the equality are neutral terms, then applying the $\eta$ rule won't change anything.
Therefore, we only have to apply the function $\eta$ rule in cases where at least one side of the equality is a function literal, leading us to defining our two $\eta$ rules.

\subsection{Type Directed Equality}

\begin{figure}[!htb]
  \fbox{$\tyEqJ{\Gamma}{n_a}{n_b}{n_t}$}
  \fbox{$\chkEqJ{\Gamma}{n_a}{n_b}{n_t}$}
  \fbox{$\synEqJ{\Gamma}{\nu_a}{\nu_b}{n_t}$}

  \begin{mathpar}
    \inferrule*
    { \chkEqJ{\Gamma}{n_a}{n_b}{n_t}
    }
    { \tyEqJ{\Gamma}{n_a}{n_b}{n_t}
    }

    \inferrule*[left=Fun-$\eta$]
    { \freshJ{\Gamma}{y} \\
      \steps{\appE{n_a}{y}}{n_a'} \\
      \steps{\appE{n_b}{y}}{n_b'} \\
      \steps{\subst{e_c}{x}{y}}{n_c} \\
      \chkEqJ{\Gamma, y : n_d}{n_a'}{n_b'}{n_c}
    }
    { \chkEqJ{\Gamma}{n_a}{n_b}{\piE{x}{n_d}{e_c}}
    }

    \inferrule*[left=$\Pi{=}$]
    { \chkEqJ{\Gamma}{n_d}{n_d'}{\univE} \\
      \freshJ{\Gamma}{y} \\
      \steps{\subst{e_c}{x}{y}}{n_c} \\
      \steps{\subst{e_c'}{x'}{y}}{n_c'} \\
      \chkEqJ{\Gamma}{n_c}{n_c'}{\univE}
    }
    { \chkEqJ{\Gamma}{\piE{x}{n_d}{e_c}}{\piE{x'}{n_d'}{e_c'}}{\univE}
    }

    \inferrule*[left=$\univE{=}$]
    {
    }
    { \chkEqJ{\Gamma}{\univE}{\univE}{\univE}
    }

    \inferrule*[left=Conv${=}$]
    { \synEqJ{\Gamma}{\nu_a}{\nu_b}{n_t'}
    }
    { \chkEqJ{\Gamma}{\nu_a}{\nu_b}{n_t}
    }

    \inferrule*[left=Var${=}$]
    { (x : n_t) \in \Gamma
    }
    { \synEqJ{\Gamma}{x}{x}{n_t}
    }

    \inferrule*[left=App${=}$]
    { \synEqJ{\Gamma}{\nu_f}{\nu_f'}{\piE{x}{n_d}{e_c}} \\
      \chkEqJ{\Gamma}{n_a}{n_a'}{n_d} \\
      \steps{\subst{e_c}{x}{n_a}}{n_c}
    }
    { \synEqJ{\Gamma}{\appE{\nu_f}{n_a}}{\appE{\nu_f'}{n_a'}}{n_c}
    }
  \end{mathpar}
  \caption{Type Directed Equality}
  \label{fig:base-type-directed-equality}
\end{figure}

In \autoref{fig:base-type-directed-equality} we define the typed equality judgment in terms of the equality check judgment $\chkEqJ{\Gamma}{n_a}{n_b}{n_t}$ and the equality synth judgment $\synEqJ{\Gamma}{n_a}{n_b}{n_t}$.
The structure of these two judgments closely follows the structure of the synth and check typing judgments from \autoref{fig:base-typing-rules}.
Like the type check judgment, the equality check judgment can be decided when all of $\Gamma$, $n_a$, $n_b$, and $n_t$ are provided.
And similarily, given $\Gamma$, $n_a$, and $n_b$, it can be decided if there exists an $n_t$ such that the equality synth judgment holds.

We use two tricks to decide these rules.
Firsly, we always start comparing normal forms using the equality check judgment.
This way we always have the type we are comparing at available, so we can apply $\eta$ rules based on it.
Secondly, the places in which normal forms appear in neutral terms (e.g. the argument of a function application) correspond to places which use the check typing judgment during type checking.
Therefore, we can infer the type of these neutral terms and then apply the equality check judgment to them, as required by our first trick.

In fact, most of the type-directed equality rules follow the same structure as their corresponding bidirectional typing rules.
For example, the App= rule corresponds to the $\Pi$-E typing rule from \autoref{fig:base-typing-rules} and follows the same structure where synth judgment is applied to the function, and then the check judgment is applied to the argument.
However, the Fun-$\eta$ rule differs from this pattern.
Instead of following the structure of the $\Pi$-I typing rule, it uses the function extensionality principle to convert the problem of equality at the dependent function type into that of equality at the codomain type.

Therefore, we can reuse the Pfenning bidirectional typing recipe \citep{Dunfield2021} to derive the type-directed equality rules.
The only change we make is to replace the introduction rules for types with an $\eta$ rule we want to decide with the $\eta$ rule.

\subsection{Unit and Dependent Pair Types}

Up until this point, the syntax and type-directed equalities have really been the same equality but presented in two different ways.
This is because the $\eta$ rule for dependent functions can be decided both by the syntax and type-directed equalities.
However, once we add the Unit type this changes, since the $\eta$ rule for the unit type is not decidable by the syntax-directed equality.
Additionally, we add the dependent pair type to further illustrate how the type-directed equality rules closely follow from to the bidirectional typing rules.

\begin{figure}[!htb]
  \begin{displaymath}
    \begin{array}{lrcll}
      \text{Expressions} & e & \bnfdef & ... & \\
      & & \bnfalt & \pairE{e_f}{e_s} & \\
      & & \bnfalt & \fstE{e} & \\
      & & \bnfalt & \sndE{e} & \\
      & & \bnfalt & \sigmaE{x}{e_f}{e_s} & \\
      & & \bnfalt & \unitE & \\
      & & \bnfalt & \unittE & \\
      \\
      \text{Normal Forms} & n & \bnfdef & ... & \\
      & & \bnfalt & \pairE{n_f}{n_s} & \\
      & & \bnfalt & \sigmaE{x}{n_f}{e_s} & \\
      & & \bnfalt & \unitE & \\
      & & \bnfalt & \unittE & \\
      \\
      \text{Neutral Terms} & \nu & \bnfdef & ... & \\
      & & \bnfalt & \fstE{\nu} & \\
      & & \bnfalt & \sndE{\nu} & \\
    \end{array}
  \end{displaymath}
  \caption{Unit and Dependent Pair Syntax}
  \label{fig:unit-dependent-pair-syntax}
\end{figure}

In \autoref{fig:unit-dependent-pair-syntax} we present the extension of the syntax with the unit and dependent pair types.
The Unit type has type constructor Unit, and introduction form $\unitE$, while the dependent pair type has type former $\sigmaE{x}{e_f}{e_s}$, introduction form $\pairE{e_f}{e_s}$, and elimination forms $\fstE{e}$ and $\sndE{e}$.

\begin{figure}[!htb]
  \begin{mathpar}
    \inferrule*[left=$\Sigma$-I$\Downarrow$]
    { \steps{e_f}{n_f} \\
      \steps{e_s}{n_s} \\
    }
    { \steps{\pairE{e_f}{e_s}}{\pairE{n_f}{n_s}}
    }

    \inferrule*[left=$\sigma_1$]
    { \steps{e}{\pairE{n_f}{n_s}} \\
    }
    { \steps{\fstE{e}}{n_f}
    }

    \inferrule*[left=$\sigma_2$]
    { \steps{e}{\pairE{n_f}{n_s}} \\
    }
    { \steps{\sndE{e}}{n_s}
    }

    \inferrule*[left=$\sigma$-I1$\Downarrow$]
    { \steps{e}{\nu}
    }
    { \steps{\fstE{e}}{\fstE{\nu}}
    }

    \inferrule*[left=$\sigma$-I2$\Downarrow$]
    { \steps{e}{\nu}
    }
    { \steps{\sndE{e}}{\sndE{\nu}}
    }

    \inferrule*[left=$\sigma$-T$\Downarrow$]
    { \steps{e_1}{n_1}
    }
    { \steps{\sigmaE{x}{e_1}{e_2}}{\sigmaE{x}{n_1}{e_2}}
    }

    \inferrule*[left=$\unittE$-I$\Downarrow$]
    {
    }
    { \steps{\unitE}{\unitE}
    }

    \inferrule*[left=$\unittE$-T$\Downarrow$]
    {
    }
    { \steps{\unittE}{\unittE}
    }
  \end{mathpar}
  \caption{Unit and Dependent Pair Semantics}
  \label{fig:unit-dependent-pair-semantics}
\end{figure}
The semantics presented for the unit and dependent pairs in \autoref{fig:unit-dependent-pair-semantics} is completely standard, as are the typing rules presented in \autoref{fig:unit-dependent-pair-typing}.
\begin{figure}[!htb]
  \begin{mathpar}
    \inferrule*[left=$\Sigma$-I]
    { \checkJ{\Gamma}{e_f}{n_1} \\
      \steps{\subst{e_2}{x}{e_s}}{n_2} \\
      \checkJ{\Gamma}{e_s}{n_2}
    }
    { \checkJ{\Gamma}{\pairE{e_f}{e_s}}{\sigmaE{x}{n_1}{e_2}}
    }

    \inferrule*[left=$\Sigma$-E1]
    { \synthJ{\Gamma}{e}{\sigmaE{x}{n_1}{e_2}} \\
    }
    { \synthJ{\Gamma}{\fstE{e}}{n_1}
    }

    \inferrule*[left=$\Sigma$-E2]
    { \synthJ{\Gamma}{e}{\sigmaE{x}{n_1}{e_2}} \\
      \steps{\subst{e_2}{x}{\fstE{e}}}{n_2}
    }
    { \synthJ{\Gamma}{\sndE{e}}{n_2}
    }

    \inferrule*[left=$\Sigma$-T]
    { \checkJ{\Gamma}{e_1}{\univE} \\
      \steps{e_1}{n_1} \\
      \checkJ{\Gamma, x : n_1}{e_2}{\univE} \\
    }
    { \checkJ{\Gamma}{\sigmaE{x}{e_1}{e_2}}{\univE}
    }

    \inferrule*[left=$\unittE$-I]
    {
    }
    { \checkJ{\Gamma}{\unitE}{\unittE}
    }

    \inferrule*[left=$\unittE$-T]
    {
    }
    { \checkJ{\Gamma}{\unittE}{\univE}
    }
  \end{mathpar}
  \caption{Unit and Dependent Pair Typing}
  \label{fig:unit-dependent-pair-typing}
\end{figure}

\begin{figure}[!htb]
  \begin{mathpar}
    \inferrule*[left=$\Sigma$-$\eta$-L]
    { \steps{\fstE{n}}{n_f'} \\
      \steps{\sndE{n}}{n_s'} \\
      \stxEqJ{\Theta}{n_f}{n_f'} \\
      \stxEqJ{\Theta}{n_s}{n_s'}
    }
    { \stxEqJ{\Theta}{\pairE{n_f}{n_s}}{n}
    }

    \inferrule*[left=$\Sigma$-$\eta$-R]
    { \steps{\fstE{n}}{n_f'} \\
      \steps{\sndE{n}}{n_s'} \\
      \stxEqJ{\Theta}{n_f}{n_f'} \\
      \stxEqJ{\Theta}{n_s}{n_s'}
    }
    { \stxEqJ{\Theta}{n}{\pairE{n_f}{n_s}}
    }

    \inferrule*[left=$\Sigma$-T${=}$]
    { \stxEqJ{\Theta}{n_1}{n_1'} \\
      \freshJ{\Theta}{y} \\
      \steps{\subst{e_2}{x}{y}}{n_2} \\
      \steps{\subst{e_2'}{x'}{y}}{n_2'} \\
      \stxEqJ{\Theta}{n_2}{n_2'}
    }
    { \stxEqJ{\Theta}{\sigmaE{x}{n_1}{e_2}}{\sigmaE{x'}{n_1'}{e_2'}}
    }

    \inferrule*[left=$\Sigma$-E1${=}$]
    { \stxEqJ{\Theta}{\nu}{\nu'}
    }
    { \stxEqJ{\Theta}{\fstE{\nu}}{\fstE{\nu'}}
    }

    \inferrule*[left=$\Sigma$-E2${=}$]
    { \stxEqJ{\Theta}{\nu}{\nu'}
    }
    { \stxEqJ{\Theta}{\sndE{\nu}}{\sndE{\nu'}}
    }

    \inferrule*[left=$\unittE$-I${=}$]
    {
    }
    { \stxEqJ{\Theta}{\unitE}{\unitE}
    }

    \inferrule*[left=$\unittE$-T${=}$]
    {
    }
    { \stxEqJ{\Theta}{\unittE}{\unittE}
    }
  \end{mathpar}
  \caption{Unit and Dependent Pair Syntax Directed Equality}
  \label{fig:unit-dependent-pair-syntax-directed-equality}
\end{figure}

In \autoref{fig:unit-dependent-pair-syntax-directed-equality} we present the syntax-directed equality for the unit and dependent pairs.
We use a trick to decide the $\eta$ rule for dependent pairs similar to that used to syntactially decide the $\eta$ rule for dependent functions.
However, the same trick doesn't work for the unit $\eta$ rule.
In the case of dependent functions and pairs, if the terms being compare were both neutral, then applying the corresponding $\eta$ rule doesn't make a difference.
But, for the unit type it does make a difference, since we can judge any two neutrals equal using the unit $\eta$ rule.
Therefore, in general, we have to have the type of the two terms available to know when we should apply the unit $\eta$ rule.

\begin{figure}[!htb]
  \begin{mathpar}
    \inferrule*[left=$\Sigma$-$\eta$]
    { \steps{\fstE{n}}{n_f} \\
      \steps{\fstE{n'}}{n_f'} \\
      \chkEqJ{\Gamma}{n_f}{n_f'}{n_1} \\
      \steps{\subst{e_2}{x}{n_f}}{n_2} \\
      \steps{\sndE{n}}{n_s} \\
      \steps{\sndE{n'}}{n_s'} \\
      \chkEqJ{\Gamma}{n_s}{n_s'}{n_2}
    }
    { \chkEqJ{\Gamma}{n}{n'}{\sigmaE{x}{n_1}{e_2}}
    }

    \inferrule*[left=$\Sigma$-T${=}$]
    { \chkEqJ{\Gamma}{n_1}{n_1'}{\univE} \\
      \freshJ{\Gamma}{y} \\
      \steps{\subst{e_2}{x}{y}}{n_2} \\
      \steps{\subst{e_2'}{x'}{y}}{n_2'} \\
      \chkEqJ{\Gamma}{n_2}{n_2'}{\univE}
    }
    { \chkEqJ{\Gamma}{\sigmaE{x}{n_1}{e_2}}{\sigmaE{x'}{n_1'}{e_2'}}{\univE}
    }

    \inferrule*[left=$\Sigma$-E1${=}$]
    { \synEqJ{\Gamma}{\nu}{\nu'}{\sigmaE{x}{n_1}{e_2}}
    }
    { \synEqJ{\Gamma}{\fstE{\nu}}{\fstE{\nu'}}{n_1}
    }

    \inferrule*[left=$\Sigma$-E2${=}$]
    { \synEqJ{\Gamma}{\nu}{\nu'}{\sigmaE{x}{n_1}{e_2}} \\
      \steps{\subst{e_2}{x}{\fstE{\nu}}}{n_2}
    }
    { \synEqJ{\Gamma}{\sndE{\nu}}{\sndE{\nu'}}{n_2}
    }

    \inferrule*[left=$\unittE$-$\eta$]
    {
    }
    { \chkEqJ{\Gamma}{n}{n'}{\unittE}
    }

    \inferrule*[left=$\unittE$-T${=}$]
    {
    }
    { \chkEqJ{\Gamma}{\unittE}{\unittE}{\univE}
    }
  \end{mathpar}
  \caption{Unit and Dependent Pair Type Directed Equality}
  \label{fig:unit-dependent-pair-type-directed-equality}
\end{figure}

Finally, in \autoref{fig:unit-dependent-pair-type-directed-equality} we present the type-directed equality rules for the unit and dependent pair types.
Here we decide the unit and dependent pair $\eta$ rules in much the same way as we do for the dependent function type.
If the type we are comparing at is a unit or dependent pair, simply apply the corresponding $\eta$ and continue.


